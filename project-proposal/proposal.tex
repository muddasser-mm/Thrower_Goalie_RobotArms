\documentclass[12pt,pdftex,a4paper]{article}
\usepackage[ngerman]{babel}
\usepackage{amsmath}
\usepackage{amssymb}
\usepackage{bbm}
\newcommand{\bbN}{\mathbbm{N}}
\newcommand{\bbR}{\mathbbm{R}}
\newcommand{\bbZ}{\mathbbm{Z}}
\newcommand{\bbI}{\mathbbm{I}}
\usepackage[pdftex]{graphicx}
\usepackage{listings}
\lstset{language=Python,basicstyle=\footnotesize}
\begin{document}
\title{Practical Course Robotics\\Project Proposal}
\author{Mohammed Muddasser\\Raphael Brösamle}
\date{}
\maketitle
%%%%%%%%%%%%%%%%%%%%%%%%%%%%%%%%%%%%%%%%%%%%%%%%
\section*{Objective}

\newpage
\section*{Work Plan}
Our work plan can be structured into four work packages.

\subsection*{Work Package 1: Throwing the ball}
We will be using a robot to throw balls at the target.
We first have to bring the robot into the correct position to be able to grasp the ball. 
Then we have to grasp the ball and lift it up.
We have to apply a force to the ball to increase its velocity.
And finally we have to open the gripper to let the ball fly away. \\
$ $ \\
Some of these steps might be a bit difficulty to realize.
We have to apply the right amount of force in the correct direction so the thrown ball will reach the its destination.
This can be done by trial and error, since the ball does not need to hit the target at a specific point. \\
During some initial trials we recognized that the default gripper can not properly grasp the ball.
When we tried to throw the ball in some random direction the ball slips out of the robots fingers.
To fix this issue we have to either improve the throwing technique or change the robot topology.
If we replace the two fingers of the robot with some kind of cup, then this issue might be solved. \\
$ $ \\
We estimate that this work package will take approximately ...

\subsection*{Work Package 2: Ball Management}
The Ball Management will manage all balls in the simulation.
It will be used to add balls to and remove balls from the simulation.
It will also be used to calculate the trajectory for each ball.
The goal keeper will be using this trajectory calculation to stop the balls. \\
$ $ \\
We estimate that this work package will take approximately ...

\subsection*{Work Package 3: Goal keeper}
The Goal keeper will be a robot that will have some kind of glove instead of its two fingers.
It will be using the trajectory calculation to know where the balls will be hitting the goal.
Then it will move the glove to the calculated position of the ball that will hit the goal first. \\
$ $ \\
We estimate that this work package will take approximately ...

\subsection*{Work Package 4: Additional stuff}
There is some additional stuff that needs to be setup.
A goal must be created and positioned, a plane where balls that have not yet been thrown will laying must be created and positioned.
The Goal keeper and all ball throwing robots will have to be correctly positioned. \\
$ $ \\
We estimate that this work package will take approximately ...

%%%%%%%%%%%%%%%%%%%%%%%%%%%%%%%%%%%%%%%%%%%%%%%%
\end{document}

